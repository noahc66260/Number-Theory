\documentclass[12 pt]{amsart}

\usepackage{amssymb}
\usepackage{latexsym}
\usepackage{graphicx}
\usepackage{verbatim}
\usepackage{enumerate}
\usepackage{amsmath}
\usepackage{fullpage}
\usepackage{rotating}
\usepackage{anyfontsize}

\begin{document}

\allowdisplaybreaks
\title
[Problem Set 7]
{Problem Set 7 \\
MATH 115 \\
Number Theory \\
Professor Paul Vojta}

\author{Noah Ruderman}

\date{ August 11, 2014}

\maketitle
\begin{center}
	Problems 7.1.1, 7.1.4, 7.1.5, 7.3.3abd, 7.3.5, 7.4.1, 7.4.3, 7.4.4, 7.4.7, 7.5.1, 7.5.3, 7.5.4, 7.6.4, and 7.6.5 
	from \emph{An Introduction to The Theory of Numbers}, 
	$5^{\text{th}}$ edition,
	by Ivan Niven, Herbert S. Zuckerman, and Hugh L. Montgomery 
\end{center}

%\vfill
%\begin{center}
%  \begin{turn}{30}
%  \fontsize{2.5cm}{2cm}\selectfont
%    \textbf{DRAFT}
%  \end{turn}
%\end{center}
%\vfill

% PROBLEM 1
\newpage
\phantom{\quad} \vfill
\noindent
\textbf{Problem} (7.1.1) \\[4ex]
  Expand the rational fractions
  $17/3$,
  $3/17$,
  and
  $8/1$
  into finite simple continued fractions.
  \\[2ex]
\emph{Solution.} \\[2ex]
  \indent
  \textbf{We start with $\frac{17}{3}$.}
  Using the Euclidean algorithm, we see that 
  \begin{align*}
    17 &= 5 \cdot 3 + 2 \\
    3 &= 1 \cdot 2 + 1 \\
    2 &= 2 \cdot 1 + 0
  \end{align*}
  So we see that $a_0 = 5$, $a_1 = 1$ and $a_2 = 2$.
  Thus, we get
  \begin{align*}
    \frac{17}{3} 
    &= 
     \langle 5, 1, 2 \rangle \\
    &=
     5 + \frac{1}{1 + \frac{1}{2}}.
  \end{align*}

  \textbf{Next we have $\frac{3}{17}$.}
  Using the Euclidean algorithm, we see that 
  \begin{align*}
    3 &= 0 \cdot 17 + 3 \\
    17 &= 5 \cdot 3 + 2 \\
    3 &= 1 \cdot 2 + 1 \\
    2 &= 2 \cdot 1 + 0
  \end{align*}
  So we see that $a_0 = 0$, $a_1 = 5$, $a_2 = 1$ and $a_3 = 2$.
  Thus, we get
  \begin{align*}
    \frac{3}{17} 
    &=
     \langle 0, 5, 1, 2 \rangle \\
    &= 
     \frac{1}{5 + \frac{1}{1 + \frac{1}{2}}}.
  \end{align*}

  \textbf{Finally we have $\frac{8}{1}$.}
  Using the Euclidean algorithm, we see that 
  \begin{align*}
    8 
    &=
      \langle 8 \rangle \\
    &=
      8 \cdot 1 + 0.
  \end{align*}
  We see that $a_0 = 8$.
  Thus, we get
  \begin{align*}
    8 &= 8
  \end{align*}
  \qed 
\vfill
\newpage



% PROBLEM 2
\phantom{\quad} \vfill
\noindent
\textbf{Problem} (7.1.4) \\[4ex]
  Given positive integers $b$, $c$, $d$ with 
  $c > d$, prove that 
  $\langle a,c \rangle < \langle a, d \rangle$ but
  $\langle a,b,c \rangle > \langle a,b, d \rangle$
  for any integer $a$.
  \\[2ex]
\emph{Solution.} \\[2ex]
  We see that 
  \begin{align*}
    c &> d \\
    \frac{1}{d} &> \frac{1}{c} & d,c \in \mathbb{Z}^+\\
    a + \frac{1}{d} &> a + \frac{1}{c} \\
    \langle a,d \rangle &> \langle a, c \rangle,
  \end{align*}
  proving the first part.

  Next we see that 
  \begin{align*}
    c &> d \\
    \frac{1}{d} &> \frac{1}{c} & d,c \in \mathbb{Z}^+ \\
    b + \frac{1}{d} &> b + \frac{1}{c} \\
    \langle b,d \rangle &> \langle b, c \rangle \\
    \frac{1}{\langle b,c \rangle} &> \frac{1}{\langle b, d \rangle} \\
    a + \frac{1}{\langle b,c \rangle} &> a + \frac{1}{\langle b, d \rangle} \\
    \langle a,b,c \rangle &> \langle a,b, d \rangle,
  \end{align*}
  proving the second part.
  \qed
\vfill
\newpage



% PROBLEM 3
\phantom{\quad} \vfill
\noindent
\textbf{Problem} (7.1.5) \\[4ex]
  Let $a_1$, $a_2$, $\cdots$, $a_n$ and $c$
  be positive real numbers.
  Prove that 
  \[
    \langle a_0, a_1, \cdots, a_n \rangle 
    > 
    \langle a_0, a_1, \cdots, a_n + c \rangle 
  \]
  holds if $n$ is odd, but is false if $n$ is even.
  \\[2ex]
\emph{Solution.} \\[2ex]
  Let $m \in \mathbb{Z}^+$ and suppose
  $2 \leq m \leq n$.
  We use induction to show that if
  \[
    \langle a_m, a_{m+1} \cdots, a_n \rangle 
    > 
    \langle a_m, a_{m+1} \cdots, a_n + c\rangle 
  \]
  then
  \[
    \langle a_{m-2}, a_{m-1}, \cdots, a_n \rangle 
    > 
    \langle a_{m-2}, a_{m-1}, \cdots, a_n + c\rangle.
  \]
  Suppose 
  \[
    \langle a_m, a_{m+1} \cdots, a_n \rangle 
    > 
    \langle a_m, a_{m+1} \cdots, a_n + c\rangle 
  \]
  We see that 
  \begin{align}
    \notag
    \langle a_m, a_{m+1} \cdots, a_n \rangle 
    &> 
      \langle a_m, a_{m+1} \cdots, a_n + c\rangle  \\
    \notag
    \frac{1}{\langle a_m, a_{m+1} \cdots, a_n + c\rangle}
    &> 
      \frac{1}{\langle a_m, a_{m+1} \cdots, a_n\rangle}  \\
    \notag
    a_{m-1} + \frac{1}{\langle a_m, a_{m+1} \cdots, a_n + c\rangle}
    &> 
      a_{m-1} + \frac{1}{\langle a_m, a_{m+1} \cdots, a_n\rangle}  \\
    \label{eq:7.1.5.1}
    \langle a_{m-1}, a_{m} \cdots, a_n + c\rangle 
    &> 
      \langle a_{m-1}, a_{m} \cdots, a_n \rangle  \\
    \notag
    \frac{1}{\langle a_{m-1}, a_{m} \cdots, a_n\rangle} 
    &> 
      \frac{1}{\langle a_{m-1}, a_{m} \cdots, a_n + c\rangle}  \\
    \notag
    a_{m-2} + \frac{1}{\langle a_{m-1}, a_{m} \cdots, a_n\rangle} 
    &> 
      a_{m-2} + \frac{1}{\langle a_{m-1}, a_{m} \cdots, a_n + c\rangle}  \\
    \label{eq:7.1.5.2}
    \langle a_{m-2}, a_{m-1}, \cdots, a_n \rangle 
    &> 
      \langle a_{m-2}, a_{m-1}, \cdots, a_n + c\rangle,
  \end{align}
  completing the induction step.
  It should be clear that this induction step also holds if 
  we replace the $>$ sign with $<$.

  We see that 
  $\langle a_n \rangle = a_n < a_n + c = \langle a_n + c \rangle$.
  We have two cases
  \begin{enumerate}
  \item $n$ is even. \\
    If $n = 0$, then 
    $\langle a_0 \rangle = a_0 < a_0 + c = \langle a_0 + c \rangle$,
    and we are done.
    If $n > 0$, then we can use the above result to get
    $\langle a_{n-2}, a_{n-1}, a_n \rangle 
    < \langle a_{n-2}, a_{n-1}, a_n + c \rangle$.
    Clearly, $n - 2$ and $n$ have the same parity.
    We can repeat this induction step until it terminates
    where we get
    \[
      \langle a_0, a_1, \cdots, a_n \rangle 
      < 
      \langle a_0, a_1, \cdots, a_n + c \rangle 
    \]
  \item $n$ is odd. \\
    If $n = 1$, then we use 
    $\langle a_1 \rangle = a_1 < a_1 + c = \langle a_1 + c \rangle$
    and equation \ref{eq:7.1.5.1} with $m = n = 1$ to get
    $\langle a_0, a_1 + c \rangle < \langle a_0, a_1 \rangle$ and 
    we are done.
    If $n > 1$, then using equation \ref{eq:7.1.5.1} with 
    $m = n$ and $\langle a_n \rangle < \langle a_n + c \rangle$
    to get
    $\langle a_{n-1}, a_n + c\rangle < \langle a_{n-1}, a_n \rangle$.
    Now we can use the induction step to get
    $\langle a_{n-3}, \ldots, a_n + c\rangle 
    < \langle a_{n-1}, \ldots, a_n \rangle$,
    and so on until it terminates. 
    Clearly, $n-1$ and $n-3$ have the same parity.
    When the induction terminates, we get
    \[
      \langle a_0, a_1, \cdots, a_n + c \rangle 
      < 
      \langle a_0, a_1, \cdots, a_n \rangle 
    \]
  \end{enumerate}
  Thus,
  \[
    \langle a_0, a_1, \cdots, a_n \rangle 
    > 
    \langle a_0, a_1, \cdots, a_n + c \rangle 
  \]
  is true if $n$ is odd, but false if $n$ is even.
  \qed
\vfill
\newpage



% PROBLEM 4
\phantom{\quad} \vfill
\noindent
\textbf{Problem} (7.3.3) \\[4ex]
  Evalue the infinite continued fractions:
	\begin{enumerate}
    \item[a.] 
       $\langle 2,2,2,2, \cdots \rangle$
    \item[b.] 
       $\langle 1,2,1,2,1,2 \cdots \rangle$
    \item[d.] 
       $\langle 1,3,1,2,1,2,1,2 \cdots \rangle$
	\end{enumerate}
\emph{Solution.} \\[2ex]
	\begin{enumerate}
		\item[a.]
      Let $x = \langle 2,2,2,2, \cdots \rangle$.
      We see that
      \begin{align*}
        x &= 2 + \frac{1}{x} \\
        x^2 &= 2x + 1 \\
        x^2 - 2x - 1 &= 0.
      \end{align*}
      From here we can use the general solution to 
      quadratic equations to get
      \begin{align*}
        x &= \frac{2 \pm \sqrt{4 + 4}}{2} \\
          &= 1 \pm \frac{\sqrt{8}}{2} \\
          &= 1 \pm \sqrt{2}
      \end{align*}
      But $x > 0$, so the only valid solution is 
      $x = 1 + \sqrt{2}$.
		\item[b.]
      Let $x = \langle 1,2,1,2, \cdots \rangle$.
      We see that
      \begin{align*}
        x &= 1 + \frac{1}{2 + \frac{1}{x}} \\
        x &= 1 + \frac{x}{2x + 1} \\
        (2x + 1)x &= (2x + 1) + x  \\
        2x^2 + x &= 2x + 1 + x \\
        2x^2 - 2x - 1 &= 0.
      \end{align*}
      From here we can use the general solution to 
      quadratic equations to get
      \begin{align*}
        x &= \frac{2 \pm \sqrt{4 + 8}}{4}\\
          &= \frac{1}{2} \pm \frac{\sqrt{12}}{4} \\
          &= \frac{1 \pm \sqrt{3}}{2}
      \end{align*}
      But $x > 0$, so the only valid solution is 
      $x = \frac{1 + \sqrt{3}}{2}$.
		\item[d.]
      Let $x = \langle 1,3,1,2,1,2, \cdots \rangle$.
      Let $y = \langle 1,2,1,2, \cdots \rangle$.
      From our results in part b, we see that 
      $y = \frac{1 + \sqrt{3}}{2}$.
      We have
      \begin{align*}
        x &= 1 + \frac{1}{3 + \frac{1}{y}} \\
        x &= 1 
             + \frac
                {1}
                {3 + \frac
                        {1}
                        {\frac
                           {1}
                           {2}
                         + \frac
                             {\sqrt{3}}
                             {2}
                        }
                } \\
        x &= 1 
             + \frac
                {1}
                {3 + \frac
                        {2}
                        {1 + \sqrt{3}}
                } \\
        x &= 1 
             + \frac
                {1 + \sqrt{3}}
                {3(1 + \sqrt{3}) + 2} \\
        x &= 1 
             + \frac
                {1 + \sqrt{3}}
                {5 + 3 \sqrt{3}} \\
        x &= 1 
             + \frac
             {(1 + \sqrt{3})(5 - 3 \sqrt{3})}
                {25 - 27} \\
        x &= 1 
             + \frac
             {(1 + \sqrt{3})(5 - 3 \sqrt{3})}{-2} \\
        x &= 1 + \frac{5 - 3 \sqrt{3} + 5 \sqrt{3} - 9}{-2}  \\
        x &= 1 + \frac{-4 + 2 \sqrt{3}}{-2}  \\
        x &= 1 + 2 - \sqrt{3}  \\
        x &= 3 - \sqrt{3}.
      \end{align*}
	\end{enumerate}
  \qed
\vfill
\newpage



% PROBLEM 5
%\phantom{\quad} \vfill
\noindent
\textbf{Problem} (7.3.5) \\[4ex]
  Let $u_0/u_1$ be a rational number in lowest terms, and write
  $u_0 / u_1 = \langle a_0, a_1, \cdots, a_n \rangle$.
  Show that if $0 \leq i < n$, then 
  $|r_i - u_0 / u_1| \leq 1 / (k_i k_{i+1})$,
  with equality if and and only if $i = n - 1$.
\\[2ex]
\emph{Solution.} \\[2ex]
  First we note that 
  \begin{align*}
    \left| r_i - \frac{u_0}{u_1} \right|
    &= 
    \left| r_i - r_n \right|
  \end{align*}
  Suppose $n = i + 1$.
  By theorem 7.5, we have
  \begin{align*}
    \left| r_i - r_n \right|
    &=
      \left| r_i - r_{i+1} \right| \\
    &=
      \left| r_{i+1} - r_{i} \right| \\
    &=
      \left| \frac{(-1)^i}{k_i k_{i+1}} \right| \\
    &=
      \frac{1}{k_i k_{i+1}},
  \end{align*}
  because $k_i \in \mathbb{Z}^+$ for all $i \in \mathbb{N}$.
  Now suppose $n - 1 > i$.
  We have four cases
  \begin{enumerate}[(1)]
    \item $n$ is odd and $i$ is even: \\
      We see that 
      \begin{align*}
        0 < r_n - r_i &< r_{n-2} - r_i \\
                      &< r_{n-4} - r_i \\
                      &\phantom{<\qquad}\vdots \\
                      &< r_{i+1} - r_i \\
                      &= \frac{1}{k_i k_{i+1}}.
      \end{align*}
      From this we can deduce that
      \[
        \frac{1}{k_i k_{i+1}} < 0 < r_n - r_i < \frac{1}{k_i k_{i+1}},
      \]
      so 
      \[
        | r_n - r_i | < \frac{1}{k_i k_{i+1}}.
      \]
    \item $n$ is even and $i$ is odd: \\
      We see that
      \begin{align*}
        0 < r_i - r_n &< r_i - r_{n-2} \\
                      &< r_i - r_{n-4} \\
                      &\phantom{<\qquad}\vdots \\
                      &< r_i - r_{i+1} \\
                      &= -(r_{i+1} - r_i) \\
                      &= - \frac{(-1)^i}{k_{i+1}k_i} \\
                      &= \frac{1}{k_i k_{i+1}}.
      \end{align*}
      From this we can deduce that
      \[
        \frac{1}{k_i k_{i+1}} < 0 < r_i - r_n < \frac{1}{k_i k_{i+1}},
      \]
      so 
      \[
        | r_n - r_i | < \frac{1}{k_i k_{i+1}}.
      \]
    \item $n$ is odd and $i$ is odd: \\
      We see that 
      \begin{align*}
        0 < r_i - r_n &< r_i - r_{n-1}.
      \end{align*}
      Call $n - 1 = n'$.
      If $n' - 1 = i$, then
      \begin{align*}
        r_i - r_{n-1} &= r_i - r_{n'} \\
                      &= r_i - r_{i+1} \\
                      &= \frac{1}{k_i k_{i+1}},
      \end{align*}
      in which case we see that 
      \[
        \frac{1}{k_i k_{i+1}} < 0 < r_i - r_n < \frac{1}{k_i k_{i+1}},
      \]
      so 
      \[
        | r_n - r_i | < \frac{1}{k_i k_{i+1}}.
      \]
      If $n' - 1 > i$, then we can appeal to case 2 to
      see that $r_i - r_{n'} < \frac{1}{k_i k_{i+1}}$ 
      so again we have
      \[
        \frac{1}{k_i k_{i+1}} < 0 < r_i - r_n < \frac{1}{k_i k_{i+1}},
      \]
      and thus
      \[
        | r_n - r_i | < \frac{1}{k_i k_{i+1}}.
      \]
    \item $n$ is even and $i$ is even: \\
      We see that 
      \begin{align*}
        0 < r_n - r_i &< r_{n-1} - r_{i}.
      \end{align*}
      Call $n - 1 = n'$.
      If $n' - 1 = i$, then
      \begin{align*}
        r_{n-1} - r_{i} &= r_{n'} - r_{i} \\
                        &= r_{i+1} - r_{i} \\
                        &< \frac{1}{k_i k_{i+1}},
      \end{align*}
      in which case we see that 
      \[
        \frac{1}{k_i k_{i+1}} < 0 < r_n - r_i < \frac{1}{k_i k_{i+1}},
      \]
      so 
      \[
        | r_n - r_i | < \frac{1}{k_i k_{i+1}}.
      \]
      If $n' - 1 > i$, then we can appeal to case 1 to
      see that $r_{n'} - r_{i} < \frac{1}{k_i k_{i+1}}$ 
      so again we have
      \[
        \frac{1}{k_i k_{i+1}} < 0 < r_n - r_i < \frac{1}{k_i k_{i+1}},
      \]
      and thus
      \[
        | r_n - r_i | < \frac{1}{k_i k_{i+1}}.
      \]
  \end{enumerate}
  \qed
\vfill
\newpage



% PROBLEM 6
\phantom{\quad} \vfill
\noindent
\textbf{Problem} (7.4.1) \\[4ex]
  Expand each of the following as infinite simple continued fractions:
  $\sqrt{2}, \sqrt{2} - 1, \sqrt{2}/2, \sqrt{3}, 1/\sqrt{3}$.
  \\[2ex]
\emph{Solution.} \\[2ex]
  \indent
  \textbf{We start with $\sqrt{2}$.}
  We see that
  \[
    \xi_0 = \sqrt{2}.
  \]
  We see that $1 < \sqrt{2} < 2$ so $a_0 = 1$.
  Next we have
  \begin{align*}
    \xi_1 &= \frac{1}{\xi_0 - a_0} \\
          &= \frac{1}{\sqrt{2} - 1} \\
          &= \sqrt{2} + 1,
  \end{align*}
  so $a_1 = 2$.
  Next we have
  \begin{align*}
    \xi_2 &= \frac{1}{\xi_1 - a_1} \\
          &= \frac{1}{\sqrt{2} + 1 - 2} \\
          &= \frac{1}{\sqrt{2} - 1} \\
          &= \sqrt{2} + 1,
  \end{align*}
  so $a_2 = 2$.
  Let $\i \in \mathbb{N}$.
  Evidently $\xi_i = \sqrt{2} + 1$ implies
  $\xi_{i+1} = \xi_i$. 
  $\xi_1 = \sqrt{2} + 1$ so $\xi_i = \xi_1$ for $i \geq 1$.
  It should be clear that if $a_i = a_1$ for all $i \geq 1$.
  From this we get
  \begin{align*}
    \sqrt{2}
    &=
    \langle 1, a_1, a_1, a_1, \ldots \rangle \\
    &=
    \langle 1, 2, 2, 2, \ldots \rangle.
  \end{align*}

  \textbf{Now we continue with $\sqrt{2} - 1$.}
  We see that
  \begin{align*}
    \sqrt{2} - 1 
    &= 
      \langle 1, 2, 2, 2, \ldots \rangle - 1\\
    &= 
      \left(
        1 + \frac{1}{\langle 2, 2, 2, \ldots \rangle}
      \right) - 1 \\
    &= 
      0 + \frac{1}{\langle 2, 2, 2, \ldots \rangle} \\
    &= 
      \langle 0, 2, 2, 2, \ldots \rangle \\
  \end{align*}

  \textbf{Next we have $\frac{\sqrt{2}}{2}$.}
  We see that $\frac{\sqrt{2}}{2} = \frac{1}{\sqrt{2}}$ and
  \begin{align*}
    \frac{1}{\sqrt{2}} 
    &=
      0 + \frac{1}{\sqrt{2}}  \\
    &=  
      0 + \frac{1}{\langle 1, 2, 2, 2, \cdots \rangle} \\
    &=  
      \langle 0, 1, 2, 2, 2, \cdots \rangle
  \end{align*}

  \textbf{We continue with $\sqrt{3}$.}
  We start with 
  \[
    \xi_0 = \sqrt{3}.
  \]
  We see that $1 < \sqrt{3} < 2$ so $a_0 = 1$.
  Next we have
  \begin{align*}
    \xi_1 &= \frac{1}{\xi_0 - a_0} \\
          &= \frac{1}{\sqrt{3} - 1} \\
          &= \frac{\sqrt{3} + 1}{2},
  \end{align*}
  so $a_1 = 1$.
  Next we have
  \begin{align*}
    \xi_2 &= \frac{1}{\xi_1 - a_1} \\
          &= \frac{1}{\frac{\sqrt{3} + 1}{2} - 1} \\
          &= \frac{1}{\frac{\sqrt{3} - 1}{2}} \\
          &= \frac{2}{\sqrt{3} - 1} \\
          &= \frac{2(\sqrt{3} + 1)}{2} \\
          &= \sqrt{3} + 1 \\
  \end{align*}
  so $a_2 = 2$.
  Next we have
  \begin{align*}
    \xi_3 &= \frac{1}{\xi_2 - a_2} \\
          &= \frac{1}{\sqrt{3} + 1 - 2} \\
          &= \frac{1}{\sqrt{3} - 1} \\
          &= \frac{\sqrt{3} + 1}{2} 
  \end{align*}
  Let $\i \in \mathbb{N}$.
  Evidently $\xi_i = \frac{\sqrt{3} + 1}{2}$ implies
  $\xi_{i+2} = \xi_i$. 
  $\xi_1 = \frac{\sqrt{3} + 1}{2}$ so $\xi_i = \xi_1$
  for $i \equiv 1 \mod 2$.
  Furthermore, $\xi_i = \xi_2$ for $i \equiv 0 \mod 2$ and 
  $i \geq 1$.
  Thus, $a_1 = a_3 = a_5 = \cdots$ and 
  $a_2 = a_4 = a_6 \cdots$, so we have
  \begin{align*}
    \sqrt{3}
    &=
    \langle a_0, a_1, a_2, a_1, a_2, \ldots \rangle \\
    &=
    \langle 1, 1, 2, 1, 2, \ldots \rangle.
  \end{align*}

  We start with 
  \[
    \xi_0 = \frac{1}{\sqrt{3}}.
  \]

  \textbf{We finish with $\frac{1}{\sqrt{3}}$.}
  We see that
  \begin{align*}
    \frac{1}{\sqrt{3}} 
    &= 
      0 + \frac{1}{\sqrt{3}} \\
    &= 
      0 + \frac{1}{\langle 1, 1, 2, 1, 2, \ldots \rangle} \\
    &= 
      \langle 0, 1, 1, 2, 1, 2, \ldots \rangle \\
  \end{align*}
  \qed
\vfill
\newpage



% PROBLEM 7
\phantom{\quad} \vfill
\noindent
\textbf{Problem} (7.4.3) \\[4ex]
  Let $\alpha, \beta, \gamma$ be irrational numbers satisfying
  $\alpha < \beta < \gamma$.
  If $\alpha$ and $\gamma$ have identical convergents
  $h_0/k_0, h_1/k_1, \cdots,$ up to $h_n/k_n$, prove that
  $\beta$ also has these same convergents up to $h_n/k_n$.
  \\[2ex]
\emph{Solution.} \\[2ex]
  Let 
  \begin{align*}
    r_i^{\alpha} &= \langle a_0, a_1, \cdots, a_i \rangle \\
    r_i^{\beta} &= \langle b_0, b_1, \cdots, b_i \rangle \\
    r_i^{\gamma} &= \langle c_0, c_1, \cdots, c_i \rangle 
  \end{align*}
  We are given $r_i^{\alpha} = r_i^{\gamma}$ for
  $0 \leq i \leq n$.
  By theorem 7.1, $a_i = c_i$ for $0 \leq i \leq n$.
  We use induction to show that 
  $a_i = b_i = c_i$ for all $0 \leq i \leq n$.

  Consider the usual algorithm for calculating the 
  $n^{\text{th}}$ convergent to an irrational number, $\xi$.
  For each term $x_i$ in the 
  $\langle x_0, x_1, x_2, \cdots, x_n \rangle $, 
  we have $x_i = \left[ \xi_i \right]$,
  $\xi_{i+1} = \frac{1}{\xi_i - x_i}$ and 
  $\xi_0 = \xi$.
  Suppose 
  $\xi^{\alpha}_i < \xi^{\beta}_i < \xi^{\gamma}_i$
  and that $a_i = b_i = c_i$.
  We see that 
  \begin{align*}
    \xi_i^{\beta} &< \xi_i^{\gamma} \\
    \xi_i^{\beta} - b_i &< \xi_i^{\gamma} - c_i\\
    \frac{1}{\xi_i^{\gamma} - c_i} &< \frac{1}{\xi_i^{\beta} - b_i}\\
    \xi_{i+1}^{\gamma} &< \xi_{i+1}^{\beta}.
  \end{align*}
  Likewise, we see that
  \begin{align*}
    \xi_i^{\alpha} &< \xi_i^{\beta} \\
    \xi_i^{\alpha} - a_i &< \xi_i^{\beta} - b_i\\
    \frac{1}{\xi_i^{\beta} - b_i} &< \frac{1}{\xi_i^{\alpha} - a_i}\\
    \xi_{i+1}^{\beta} &< \xi_{i+1}^{\alpha}.
  \end{align*}
  Thus we have 
  \[
    \xi_{i+1}^{\gamma} < \xi_{i+1}^{\beta} < \xi_{i+1}^{\alpha}.
  \]
  Furthermore, we see that 
  \begin{align*}
    \xi_{i+1}^{\gamma} &< \xi_{i+1}^{\beta} < \xi_{i+1}^{\alpha} \\
    \left[ \xi_{i+1}^{\gamma} \right] 
      &\leq \left[ \xi_{i+1}^{\beta} \right] 
      \leq \left[ \xi_{i+1}^{\alpha} \right] \\
    c_{i+1} &\leq b_{i+1} \leq a_{i+1}.
  \end{align*}
  But $a_{i+1} = c_{i+1}$ for $0 \leq i+1 \leq n$, so 
  \[
    a_{i+1} = b_{i+1} =  c_{i+1}.
  \]

  We  have proved 
  $\xi^{\alpha}_i < \xi^{\beta}_i < \xi^{\gamma}_i$
  and $a_i = b_i = c_i$ implies
  $\xi^{\gamma}_{i+1} < \xi^{\beta}_{i+1} < \xi^{\alpha}_{i+1}$
  and $a_{i+1} = b_{i+1} =  c_{i+1}$.
  A similar argument shows 
  $\xi^{\gamma}_i < \xi^{\beta}_i < \xi^{\alpha}_i$
  and $a_i = b_i = c_i$ implies
  $\xi^{\alpha}_{i+1} < \xi^{\beta}_{i+1} < \xi^{\gamma}_{i+1}$
  and $a_{i+1} = b_{i+1} =  c_{i+1}$
  for $0 \leq i < n$.

  As our base case, we see that 
  \begin{align*}
    \alpha &< \beta < \gamma \\
    \xi_0^{\alpha} &< \xi_0^{\beta} < \xi_0^{\gamma} \\
    [\xi_0^{\alpha}] &\leq [\xi_0^{\beta}] \leq [\xi_0^{\gamma}] \\
    a_0 &\leq b_0 \leq c_0.
  \end{align*}
  Since $a_0 = c_0$, we see that $a_0 = b_0 = c_0$.

  By the induction hypothesis, we see that 
  $a_i = b_i = c_i$ for all $0 \leq i \leq n$.
  From this we see that
  \[
    \langle a_0, a_1, \cdots, a_i \rangle
    = 
    \langle b_0, b_1, \cdots, b_i \rangle
    = 
    \langle c_0, c_1, \cdots, c_i \rangle,
  \]
  so 
  \[
    r_i^{\alpha} = r_i^{\beta} = r_i^{\gamma},
  \]
  and the convergents are equal for $0 \leq i \leq n$.
  \qed
\vfill
\newpage



% PROBLEM 8
\phantom{\quad} \vfill
\noindent
\textbf{Problem} (7.4.4) \\[4ex]
  Let $\xi$ be an irrational number with continued fraction expansion
  $\langle a_0, a_1, a_2, a_3, \cdots \rangle$.
  Let $b_1, b_2, b_3, \cdots$ be any finite or infinite sequence
  of positive integers. 
  Prove that 
  \[
    \lim_{n \to \infty}
    \langle a_0, a_1, a_2, \cdots, a_n,
            b_1, b_2, b_3, \cdots \rangle
    = \xi.
  \]
  \\[2ex]
\emph{Solution.} \\[2ex]
  Let $x_n = 
  \langle a_0, a_1, a_2, \cdots, a_n, b_1, b_2, b_3, \cdots \rangle$.
  Let $r_n$ be the $n^{\text{th}}$ convergent %to $\xi$. 
  of $x_n$.
  We know that even convergents form a monotonically
  increasing sequence whose limit is $x_n$.
  Likewise, the odd convergents form a monotonically 
  decreasing sequence whose limit is $x_n$.
  Thus, if $n$ is odd, then 
  \begin{align*}
    r_{n-1} < x_n < r_n
  \end{align*}
  and if $n$ is even, then
  \begin{align*}
    r_{n} < x_n < r_{n-1}.
  \end{align*}
  But $r_n = \langle a_0, a_1, a_2, \cdots, a_n \rangle$, which
  is also the $n^{\text{th}}$ convergent of $\xi$.
  Thus, we see that 
  \begin{align*}
    \lim_{\substack{n \to \infty \\ \text{$n$ even}}}
    r_n \leq 
    \lim_{\substack{n \to \infty}}
    x_n &\leq
    \lim_{\substack{n \to \infty \\ \text{$n$ odd}}}
    r_n \\
    \xi \leq \lim_{n \to \infty} x_n &\leq \xi \\
    \lim_{n \to \infty} x_n &= \xi,
  \end{align*}
  completing the proof.
  \qed
%  First we note that 
%  \begin{align*}
%    \langle b_1, b_2, b_3, \cdots \rangle 
%    &=
%      b_1 + \frac{1}{\langle b_2, b_3, b_4 \cdots \rangle } \\
%    &> 1,
%  \end{align*}
%  so 
%  \[
%    \frac{1}{\langle b_1, b_2, b_3, \cdots \rangle} < 1.
%  \]
%  We see that 
%  \begin{align*}
%    \langle a_0, a_1, a_2, \cdots, a_n,
%            b_1, b_2, b_3, \cdots \rangle 
%    &=
%            \left\langle a_0, a_1, a_2, \cdots, a_n 
%            + \frac{1}{\langle b_1, b_2, b_3, \cdots \rangle}
%            \right\rangle \\
%    &<
%            \left\langle a_0, a_1, a_2, \cdots, a_n + 1
%            \right\rangle.
%  \end{align*}
%  Next we use theorem 7.3 to write the $n^{\text{th}}$ convergent
%  \begin{align*}
%    r_n &= \frac{(a_n + 1)h_{n-1} + h_{n-2}}{(a_n + 1)k_{n-1} + k_{n-2}} \\
%        &< 
%          \frac{a_n h_{n-1} + h_{n-2}}{(a_n + 1)k_{n-1} + k_{n-2}}
%          + \frac{ h_{n-1} + h_{n-2}}{(a_n + 1)k_{n-1} + k_{n-2}}
%  \end{align*}

\vfill
\newpage



% PROBLEM 9
\phantom{\quad} \vfill
\noindent
\textbf{Problem} (7.4.7) \\[4ex]
  Prove that 
  \[
    k_n | k_{n-1} \xi - h_{n-1} |
    + k_{n-1} |k_n \xi - h_n |
    = 1.
  \]
  \\[2ex]
\emph{Solution.} \\[2ex]
  Suppose the above equality is true, then
  \begin{align*}
    k_n | k_{n-1} \xi - h_{n-1} |+ k_{n-1} |k_n \xi - h_n | &= 1 \\
    \left| \xi - \frac{h_{n-1}}{k_{n-1}} \right| 
    + \left| \xi - \frac{h_{n}}{k_{n}} \right| 
    &= 
      \frac{1}{k_n k_{n-1}} \\
    \left| \xi - r_{n-1} \right| 
    + \left| \xi - r_n \right| 
    &= 
      \frac{1}{k_n k_{n-1}}.
  \end{align*}
  Thus, it is sufficient to prove that 
  \begin{align*}
    \left| \xi - r_{n-1} \right| 
    + \left| \xi - r_n \right| 
    &= 
      \frac{1}{k_n k_{n-1}}.
  \end{align*}

  We know that odd convergents are larger than their limit
  and that even convergents are smaller than their limit.
  We have two cases:
  \begin{enumerate}[(1)]
    \item $n$ is even: \\
      Thus, $r_n < \xi$ and $r_{n-1} > \xi$.
      We have
      \begin{align*}
        \left| \xi - r_{n-1} \right| 
        + \left| \xi - r_n \right| 
        &=
          -(\xi - r_{n-1}) + (\xi - r_n) \\
        &=
           r_{n-1} - r_n \\
        &=
           r_n - r_{n-1} \\
        &=
           - \frac{(-1)^{n-1}}{k_n k_{n-1}} 
           & \text{by theorem 7.5}, n \geq 1 \\
        &= 
           \frac{(-1)^{n}}{k_n k_{n-1}} \\
        &= 
           \frac{1}{k_n k_{n-1}}.
      \end{align*}
    \item $n$ is odd: \\
      Thus, $r_n > \xi$ and $r_{n-1} < \xi$.
      We have
      \begin{align*}
        \left| \xi - r_{n-1} \right| 
        + \left| \xi - r_n \right| 
        &=
          (\xi - r_{n-1}) - (\xi - r_n) \\
        &=
           r_n - r_{n-1} \\
        &=
           \frac{(-1)^{n-1}}{k_n k_{n-1}} 
           & \text{by theorem 7.5}, n \geq 1\\
        &= 
           \frac{1}{k_n k_{n-1}}.
      \end{align*}
  \end{enumerate}
  In each case, we have proved a necessary condition to 
  imply the inequality. 

  However, we have used theorem 7.5 which only holds for $n \geq 1$.
  It remains to show that the equality holds for $n = 0, -1$.
  We note that $h_{-2} = k_{-1} = 0$ and $h_{-1} = k_{-2} = k_0 = 1$.
  Suppose $n = 0$. 
  Then we have
  \begin{align*}
    k_n | k_{n-1} \xi - h_{n-1} |
    + k_{n-1} |k_n \xi - h_n |
    &=
      k_0 | k_{-1} \xi - h_{-1} |
      + k_{-1} |k_0 \xi - h_0 | \\
    &=
      1 \cdot | 0 \cdot \xi - 1 |
      + 0 \cdot |k_0 \xi - h_0 | \\
    &=
      1.
  \end{align*}
  Suppose $n = -1$. 
  Then we have
  \begin{align*}
    k_n | k_{n-1} \xi - h_{n-1} |
    + k_{n-1} |k_n \xi - h_n |
    &=
      k_{-1} | k_{-2} \xi - h_{-2} |
      + k_{-2} |k_{-1} \xi - h_{-1} | \\
    &=
      0 \cdot | k_{-2} \xi - h_{-2} |
      + 1 \cdot |0 \cdot \xi - 1 | \\
    &=
      1.
  \end{align*}
  Now we have showed the equality holds for all $n \geq -1$.
  \qed
\vfill
\newpage



% PROBLEM 10
\phantom{\quad} \vfill
\noindent
\textbf{Problem} (7.5.1) \\[4ex]
  Prove that the first assertio nin theorem 7.13 holds in case
  $n = 0$ if $k_1 > 1$.
  \\[2ex]
\emph{Solution.} \\[2ex]
  We proceed in the same way that is outlined in NZM.
  Suppose the first part of theorem 7.13 if false. 
  Then 
  \begin{align*}
    \left| \xi - \frac{a}{b} \right| 
    &< 
    \left| \xi - \frac{h_n}{k_n} \right|  & b\leq k_n\\
    | \xi b - a | &< | \xi k_n - h_n |.
  \end{align*}
  Using the second part of theorem 7.13, we see that 
  this implies
  $b \geq k_{n+1}$.
  If $n = 0$, then $b \geq k_1$ and $b \leq k_0$.
  But $k_0 = 1$. 
  So if $k_1 > 1$, then $b > 1$ and $b \leq 1$, which
  is a contradiction.
  Thus, the assumption must have been false. 
  Thus, $b > k_n$ for $n = 0$ when $k_1 > 1$.
  \qed
\vfill
\newpage



% PROBLEM 11
\phantom{\quad} \vfill
\noindent
\textbf{Problem} (7.5.3) \\[4ex]
  $\ldots$ Prove that every convergent to $\xi$ is a good approximation.
  \\[2ex]
\emph{Solution.} \\[2ex]
  We use theorem 7.13 to see that 
  \begin{align*}
    | \xi b - a | < | \xi k_n - h_n | 
  \end{align*}
  For $n \geq 1$ implies $b \geq k_{n+1} > k_n$, so $b > k_n$.
  Thus, 
  \begin{align*}
    | \xi k_n - h_n | 
    &=
    \min_{\substack{\text{all $x$} \\ 0 < y \leq k_n}}
    | \xi y - x |,
  \end{align*}
  so $\frac{h_n}{k_n} = r_n$ is a "good approximation" to $\xi$.
  We proved in problem 7.5.1 that theorem 7.13 holds for $n = 0$
  when $k_1 > 1$.
  From the recursive definition of $k_i$, it is easy to see that
  $k_1 = a_1$.
  But $a_1 \in \mathbb{Z}^+$, so $a_1 \geq 1$. 
  If $a_1 > 1$, then $k_1 > 1$ and we are done so we assume that
  $k_1 = a_1 = 1$.
  Thus, 
  \begin{align*}
    \min_{\substack{\text{all $x$} \\ 0 < y \leq k_1}}
    | \xi y - x | 
    &=
    \min_{\substack{\text{all $x$} \\ 0 < y \leq 1}}
    | \xi y - x | \\
    &=
    \min_{\substack{\text{all $x$}}}
    | \xi  - x |.
  \end{align*}
  Furthermore, $h_0 = a_0 = [\xi]$, so 
  $|\xi k_0 - h_0| = |\xi - [\xi]|$.

  It actually turns out that the $0^{\text{th}}$ convergent
  is not necessarily a "good approximation".
  By theorem 4.1(1), we know that 
  $0 \leq  \xi  - [\xi] < 1$ so $|\xi - [\xi]| = \xi - [\xi]$,
  and $|\xi - [\xi] - 1| = -\xi + [\xi] + 1$.
  We see that if $\xi - [\xi] > \frac{1}{2}$, then
  \begin{align*}
    \xi - [\xi] &> \frac{1}{2} \\
    2 \xi - 2[\xi] &> 1 \\
    \xi - [\xi] &> -\xi + [\xi] + 1 \\
    |\xi - [\xi]| &> |\xi - ([\xi] + 1)| \\
    |\xi k_0 - h_0| 
    &> 
      \min_{\substack{\text{all $x$}\\ 0 < y \leq k_0}} 
      |\xi y - x|.
  \end{align*}
  Thus, every convergent $r_n$ is a "good approximation" 
  to $\xi$
  except for $r_0$ when $k_1 = 1$.
  \qed
\vfill
\newpage



% PROBLEM 12
%\phantom{\quad} \vfill
\noindent
\textbf{Problem} (7.5.4) \\[4ex]
  Prove that every "good approximation" to $\xi$ is convergent.
  \\[2ex]
\emph{Solution.} \\[2ex]
  Let $\frac{a}{b} \in \mathbb{Q}$ with $\gcd(a,b) = 1$.
  Suppose $\frac{a}{b}$ is a "good approximation" to $\xi$
  but isn't a convergent. 
  First we will show by contradiction that $b = k_i$ for some 
  $i \in \mathbb{N}$
  and then that $a = h_i$.

  If $b = k_i$, we are done, so we suppose that 
  $k_j < b < k_{j+1}$ 
  for $j \geq 1$ or $j \geq 0$ if $k_1 > 1$.
  Thus, 
  \begin{align*}
    | \xi b - a| 
    &< 
      \min_{\substack{\text{all $x$}\\0 < y \leq b}}
      | \xi y - x| \\
    | \xi b - a| 
    &< 
      | \xi k_j - h_j|,
  \end{align*}
  so $b \geq k_{j+1}$ by theorem 7.13.
  This contradicts our assumption that $b < k_{j+1}$.
  So if $\frac{a}{b}$ is a "good approximation" to $\xi$, 
  then $b = k_i$ 
  for $i \in \mathbb{N}$.

  Let $\frac{a}{b} = \frac{a}{k_i}$ be our good approximation to $\xi$.
  Using our work in problem 7.5.3, we see that the
  $i^{\text{th}}$ convergent is a good approximation to $\xi$, so
  \begin{align*}
    \min_{\substack{\text{all $x$}\\0 < y \leq b}}
    | \xi y - x| 
    &= 
      \min_{\substack{\text{all $x$}\\0 < y \leq k_i}}
      | \xi y - x| \\
    &= 
      | \xi k_i - h_i|.
  \end{align*}
  Thus, if $\frac{a}{k_i}$ is also a good approximation to $\xi$, then
  \begin{align*}
    | \xi k_i - a| &= | \xi k_i - h_i|.
  \end{align*}
  We have two cases:
  \begin{enumerate}[(1)]
    \item $\xi k_i - a = \xi k_i - h_i$: \\
      We see that 
      \begin{align*}
        | \xi k_i - a| &= | \xi k_i - h_i| \\
        \pm(\xi k_i - a) &= \pm(\xi k_i - h_i) \\
        \pm \xi k_i \mp a &= \pm \xi k_i \mp h_i \\
        \mp a &= \mp h_i \\
        a &= h_i \\
      \end{align*}
    \item $(\xi k_i - a) = -(\xi k_i - h_i)$: \\
      We see that 
      \begin{align*}
        | \xi k_i - a| &= | \xi k_i - h_i| \\
        \mp(\xi k_i - a) &= \pm(\xi k_i - h_i) \\
        \mp \xi k_i \pm a &= \pm \xi k_i \mp h_i \\
        \pm a &= \pm 2 \xi k_i \mp h_i \\
        a &= 2 \xi k_i - h_i,
      \end{align*}
      which is impossible since $a \in \mathbb{Z}$.
  \end{enumerate}
  Thus, $a = h_i$.
  In conclusion, if $\frac{a}{b}$ is a good approximation to $\xi$, 
  then $a = h_i$ and $b = k_i$ for some 
  $i \in \mathbb{N}$ 
  \qed
\vfill
\newpage



% PROBLEM 13
\phantom{\quad} \vfill
\noindent
\textbf{Problem} (7.6.4) \\[4ex]
  Given any constant $c$, prove that there exists an irrational
  number $\xi$ and infinitely many rational numbers
  $h/k$ such that 
  \begin{align}
    \label{eq:7.6.4.1}
    \left| \xi - \frac{h}{k} \right| &< \frac{1}{k^c}.
  \end{align}
  \\[2ex]
\emph{Solution.} \\[2ex]
  By theorem 7.11, we see that 
  \begin{align*}
    \left| \xi - \frac{h_n}{k_n} \right| &< \frac{1}{k_n k_{n+1}}.
  \end{align*}
  Let $c$ be an arbitrary real number.
  We see that proving $k_{n+1} \geq k_n^{c-1}$ is a sufficient
  condition to complete the proof because
  \begin{align*}
    \frac{1}{k_{n+1}} \leq \frac{1}{k_n^{c-1}} \\
    \frac{1}{k_n k_{n+1}} \leq \frac{1}{k_n^{c}}.
  \end{align*}
  By definition, $k_{n+1} = a_{n+1} k_n + k_{n-1}$.
  Thus, $a_{n+1} \geq k_n^{c-2}$ is also a sufficient condition
  because
  \begin{align*}
    k_{n+1} &= a_{n+1} k_n + k_{n-1} \\
            &\geq k_n^{c-1} + k_{n-1} \\ 
            &\geq k_n^{c-1}.
  \end{align*}
  Appealing to the recursive definition of $k_n$, we see that
  $k_n$ is a function of $a_0, a_1, \cdots, a_n$.
  Since $a_{n+1}$ is a function of $k_n$, we see that 
  it is also a function of $a_0, a_1, \cdots, a_n$.
  Thus, we can construct an infinite continued fraction 
  $\langle a_0, a_1, a_2, \cdots \rangle$ such that 
  $a_n \geq k_n^{c-2}$ for all $n \in \mathbb{N}$.
  By theorem 7.7, the value of any infinite simple continued
  fraction is irrational, so such a number
  $\xi = \langle a_0, a_1, a_2, \cdots \rangle$ is guaranteed
  to exist. 
  The rational numbers which satisfy equation 
  \ref{eq:7.6.4.1}
  are the convergents of $\xi$, of which there are 
  infinitely many.
  \qed
\vfill
\newpage



% PROBLEM 14
\phantom{\quad} \vfill
\noindent
\textbf{Problem} (7.6.5) \\[4ex]
  Prove that of every two consecutive convergents
  $h_n/k_n$ to $\xi$ with $n \geq 0$, at least one
  satisfies
  \begin{align}
    \label{eq:7.6.5.1}
    \left| \xi - \frac{h}{k} \right| &< \frac{1}{2k^2}.
  \end{align}
  \\[2ex]
\emph{Solution.} \\[2ex]
  We proceed by contradiction.
  Suppose there are two consecutive convergents
  $\frac{h_n}{k_n}$
  and 
  $\frac{h_{n+1}}{k_{n+1}}$
  for which
  \begin{align*}
    \left| \xi - \frac{h}{k} \right| > \frac{1}{2k^2}.
  \end{align*}
  Then we see that 
  \begin{align}
    \label{eq:7.6.5.2}
    \left| \xi - \frac{h_n}{k_n} \right| 
    +  
    \left| \xi - \frac{h_{n+1}}{k_{n+1}} \right| 
    > 
    \frac{1}{2k_n^2}
    +
    \frac{1}{2k_{n+1}^2}.
  \end{align}
  But $r_n < \xi < r_{n+1}$ or $r_{n+1} < \xi < r_n$ because
  odd convergents are greater than their limit and even
  convergents are less than their limit.
  We suppose that $r_{n+1} > r_n$, but note that the 
  same results follow when $r_{n+1} < r_n$. 
  We see that 
  \begin{align*}
    \left| r_{n+1} - r_n \right|
    &=
      \left| r_{n+1} - \xi + \xi - r_n \right| \\
    &=
      \left| r_{n+1} - \xi \right| + \left| \xi - r_n \right| \\
    &=
      \left| \xi - \frac{h_{n+1}}{k_{n+1}} \right| 
      + \left| \xi - \frac{h_{n}}{k_{n}} \right|.
  \end{align*}
  Furthermore, by theorem 7.5 we see that 
  \begin{align*}
    | r_{n+1} - r_n | = \frac{1}{k_{n+1} k_n}.
  \end{align*}
  Now we can plug this into equation \ref{eq:7.6.5.2} to get
  \begin{align*}
    \frac{1}{k_{n+1} k_n}
    &> 
      \frac{1}{2k_n^2}
      +
      \frac{1}{2k_{n+1}^2} \\
    2
    &> 
      \frac{k_{n+1}}{k_n} + \frac{k_n}{k_{n+1}}.
  \end{align*}

  Call $x = \frac{k_{n+1}}{k_n}$. 
  Clearly, $x > 0$.
  We wish to find solutions to the equation
  \begin{align*}
    2 &> x + x^{-1}.
  \end{align*}
  We see that 
  \begin{align*}
    2 &> x + x^{-1} \\
    2x &> x^2 + 1 & \text{no sign change as $x > 0$} \\
    0 &> x^2 - 2x + 1 \\
    0 &> (x-1)^2,
  \end{align*}
  but $(x-1)^2 > 0$ for all $x \in \mathbb{Q}^+$, so we 
  have arrived at a contradiction.
  Thus, one of the convergents
  $\frac{h_n}{k_n}$
  and 
  $\frac{h_{n+1}}{k_{n+1}}$
  must satisfy equation
  \ref{eq:7.6.5.1}.
\vfill
\qed



\end{document}
