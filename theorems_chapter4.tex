\documentclass{amsart}

\usepackage{amsthm}
\usepackage{amsmath}
\usepackage{fullpage}
\usepackage{enumerate}

\newtheorem{theorem}{Theorem}
\newtheorem{lemma}{Lemma}
\theoremstyle{definition}
\newtheorem*{definition*}{Definition}
\newtheorem*{note*}{Note}
\begin{document}

\section*{Chapter 4}

\begin{theorem}
  Let $x$ and $y$ be real numbers.
  Then
  \begin{enumerate}[(1)]
    \item
    $[x] \leq x \leq [x] + 1], x-1 < [x] \leq x, 0 \leq x - [x] < 1.$
  \item
    $[x] = \sum_{1 \leq i \leq x} 1 \quad \text{if $x \leq 0$}$.
  \item
    $[x+m] = [x] + m$ if $m$ is an integer.
  \item
    $[x] + [y] \leq [x + y] \leq [x] + [y] + 1$.
  \item
  $[x] + [-x] = \begin{cases} 0 & \text{if $x$ is an integer} \\
                              -1 & \text{otherwise} \end{cases}$
  \item
    $\left[ \frac{[x]}{m} \right] = \left[ \frac{x}{m} \right]$
    if $m$ is a positive integer.
  \item
    $\cdots$
  \end{enumerate}
  $\cdots$
\end{theorem}

\begin{theorem}
  de Policgnac's formula.
  Let $p$ denote a prime.
  Then the largest exponent $e$ such that $p^e \parallel n!$
  is 
  \[
    e = \sum^{\infty}_{i = 1} \left[ \frac{n}{p^i} \right].
  \]
\end{theorem}

\begin{definition*}
  A function $f$ is arithmetic if its domain is the positive
  integers and whose range is a subset of the complex numbers.
  In other words, $f(n)$ is defined for all positive
  integers $n$.
  \emph{Arithmetic functions} are also called
  \emph{number theoretic functions}, or
  \emph{numerical functions.}
\end{definition*}

\begin{note*}
  An arithmetic function does not need to be defined for 0.
  Also, $\phi$, or Euler's function, is an arithmetic function.
\end{note*}

\begin{definition*}
  For positive integers $n$ we make the following definitions
  \begin{itemize}
  \item[] $d(n)$ is the number of positive divisors of $n$.
  \item[] $\sigma(n)$ is the sum of the positive divisors of $n$.
  \item[] $\sigma_k(n)$ is the sum of the $k$th powers of the 
    positive divisors of $n$.
  \item[] $\omega(n)$ is the number of distinct primes dividing $n$.
  \item[] $\Omega(n)$ is the number of primes dividing $n$, counting
    multiplicity.
  \end{itemize}
\end{definition*}

\begin{note*}
  Prime numbers are positive by definition.  
\end{note*}

\begin{definition*}
  If $f(n)$ is an arithmetic function \underline{not identically zero} such
  that $f(mn) = f(m)f(n)$ for every pair of 
  \underline{positive} integers $m,n$
  satisfying $\gcd(m,n) = 1$, then $f(n)$ is said to be multiplicitive.
  If $f(mn) = f(m)f(n)$ whether $m$ and $n$ are relatively prime or not,
  then $f(n)$ is said to be totally multiplicitive or completely 
  multiplicitive. 
\end{definition*}

\begin{note*}
  If $f$ is a multiplicitive function, then $f(n) = f(n)f(1)$.
  Since there is an $n$ such that $f(n) \neq 0$, we can divide
  by $f(n)$ to reveal that $f(1) = 1$.
  Thus, an easy way to exclude a function as multiplicitive is to 
  find $f(1) \neq 1$.
\end{note*}

\begin{note*}
  For a multiplicitive function $f$, we see that for 
  $n = \prod p^{\alpha}$, we have
  $f(n) = f\left( \prod p^{\alpha} \right) = \prod f(p^{\alpha})$.
\end{note*}

\begin{theorem}
  Let $f(n)$ be a multiplicitive function and let 
  $F(n) = \sum_{d \mid n} f(d)$.
  Then $F(n)$ is multiplicitive.
\end{theorem}

\begin{theorem}
  For every positive integer $n$, 
  \begin{align*}
    \sigma(n) &= \prod_{p^{\alpha} \parallel n}
                \left( \frac{p^{\alpha + 1} - 1}{p -1} \right).
  \end{align*}
\end{theorem}

\begin{definition*}
  For positive integers $n$ put $\mu(n) = (-1)^{\omega(n)}$ 
  if $n$ is square-free, and set $\mu(n) = 0$ otherwise.
  Then $\mu(n)$ is the M\"{o}bius mu function.
\end{definition*}

\begin{theorem}
  The function $\mu(n)$ is multiplicitive and 
  \begin{align*}
    \sum_{d \mid n} \mu(d) =
    \begin{cases} 
      1 & \text{if} \quad n = 1 \\
      0 & \text{if} \quad n > 1.
    \end{cases}
  \end{align*}
\end{theorem}

\begin{theorem}
  M\"{o}bius inversion formula.
  If $F(n) = \sum_{d \mid n} f(d)$ for every positive integer $n$,
  then 
  \[
    f(n) = \sum_{d \mid n} \mu(d) F\left( \frac{n}{d} \right).
  \]
\end{theorem}

\begin{theorem}
  If $f(n) = \sum_{d \mid n} \mu(d) F\left( \frac{n}{d} \right)$ for
  every positive integer $n$, then $F(n) = \sum_{d \mid n} f(d)$.
\end{theorem}

\end{document}
