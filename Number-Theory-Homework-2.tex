\documentclass[12 pt]{amsart}

\usepackage{amssymb}
\usepackage{latexsym}
\usepackage{graphicx}
\usepackage{verbatim}
\usepackage{enumerate}
\usepackage{amsmath}
\usepackage{fullpage}
\usepackage{mathrsfs} % woa- I was able to install this myself and it works!

\begin{document}

\allowdisplaybreaks
\title
[Problem Set 2]
{Problem Set 2 \\
MATH 115 \\
Number Theory \\
Professor Paul Vojta}

\author{Noah Ruderman}

\date{ July 7, 2014}

\maketitle
\begin{center}
	Problems 1.4.4ab, 2.1.6, 2.1.12, 2.1.20, 2.1.32, 2.1.40, 2.1.59, 2.2.6abc, 
  2.2.8, 2.3.4, 2.3.7, 2.3.14, 2.3.16, 2.3.17, and 2.3.20 
	from \emph{An Introduction to The Theory of Numbers}, 
	$5^{\text{th}}$ edition,
	by Ivan Niven, Herbert S. Zuckerman, and Hugh L. Montgomery 
\end{center}

% PROBLEM 1
\clearpage
\phantom{\quad} \vfill
\noindent
\textbf{Problem} (1.4.4) \\[4ex]
\emph{Solution.} \\[2ex]
	\begin{enumerate}
		\item[a.]
      %$k \in \mathscr{S}$
      %We know that $\mathscr{S}$ has $2n$ elements.
      Consider a set $A$ with $n \in \mathbb{N}$ elements, for an even $n$. 
      Let us find a recursive expression for the number of partitions, each
      of which only containing two elements. 
      Let $P(n)$ be the number of ways we may form such partitions in a 
      finite set of $n$ elements. 
      Consider an arbitrary element $k \in A$.
      There are $(n-1)$ ways $k$ can be paired with another element, 
      and $P(n-2)$ ways the remaining $n-2$ elements can be partitioned
      into sets of 2.
      So we write
      \[
        P(n) = (n-1)\cdot P(n-2)
      \]
      Clearly, there is one way to form a partition with only 2 elements
      so 
      \[
        P(2) = 1.
      \]
      
      Now $|\mathscr{S}| = 2n$, so the number of ways we can partition
      $\mathscr{S}$ into sets of cardinality 2 is 
      \begin{align*}
        P(2n) &= (2n-1) P(2n-2) \\
              &= (2n-1)(2n-3)P(2n-4) \\
              &\ldots  \\
              &= (2n-1)(2n-3)\cdots 5 \cdot 3 \cdot  P(2) \\
              &= (2n-1)(2n-3)\cdots 5 \cdot 3 \cdot 1
      \end{align*}
      We note that 
      \begin{align}
          (2n)(2n-1)(2n-2) \cdots (2)(1) &= (2n)! \notag \\
          \left[
            (2n-1)(2n-3)\cdots (5)(3)(1)
          \right]
          \left[
            (2n)(2n-2) \cdots (6)(4)(2)
          \right]
          &= (2n)! \notag \\
          \left[
            (2n-1)(2n-3)\cdots (5)(3)(1)
          \right]
          \left[
            2^n (n)(n-1) \cdots (3)(2)(1)
          \right] 
          &= (2n)! \notag \\
          \left[
            (2n-1)(2n-3)\cdots (5)(3)(1)
          \right]
            2^n n!
          &= (2n)! \notag \\
            \label{eq:1.4.4}
          (2n-1)(2n-3)\cdots 5 \cdot 3 \cdot 1
          &= 
          \frac{(2n)!}{2^n n!}
      \end{align}
      \vfill
      \clearpage
      \phantom{\quad}
      \vfill
		\item[b.]
      It is easy to see that 
      \begin{align*}
        (n+1)(n+2) \cdots (2n) &= \frac{(2n)!}{n!}
      \end{align*}
      Using equation \ref{eq:1.4.4}, we see that 
      \begin{align*}
        \frac{(n+1)(n+2) \cdots (2n)}{2^n} 
          &= \frac{(2n)!}{2^n n!} \\
          &= (2n-1)(2n-3)\cdots 5 \cdot 3 \cdot 1 \in {N}
      \end{align*}
      so $n_0 = \prod_{k=1}^{n} (n + k)$ is divisible by $2^n$.
      Furthermore, the quotient of $\frac{n_0}{2^n}$ is odd, since
      the product of odd numbers is always odd. 
      Thus, $\frac{n_0}{2^n}$ is not divisible by 2, and thus
      $n_0$ is not divisible by $2^{n+1}$, because if it were, 
      $\frac{n_0}{2^{n+1}} = \frac{n_0 \big/ 2^n}{2}$ would be 
      an integer, which we have shown is impossible.
	\end{enumerate}
\vfill
\newpage



% PROBLEM 2
\phantom{\quad} \vfill
\noindent
\textbf{Problem} (2.1.6) \\[4ex]
\emph{Solution.} \\[2ex]
  Let $p$ be a prime number.
  If $a^2 \equiv b^2 \mod p$ then
  \begin{align*}
    a^2 &\equiv b^2 \mod p \\
    p &\mid a^2 - b^2 \\
    p &\mid (a+b) (a-b)
  \end{align*}
  By Theorem 1.15, $p$ must divide $(a+b)$ or $(a-b)$.

  %If $p$ divides $(a+b)$ we are done, so we suppose that it
  %does not.
  %Since the gcd of $p$ and another number can only be 
  %1 or $p$, we see that it must be 1 because if it were $p$, 
  %then $p \mid (a + b)$, which is false by assumption. 
  %So $\text{gcd}(p, a+b) = 1$.

  %Now we use Theorem 1.10 from Niven et al., namely
  %\begin{center}
  %  If $c' \mid a'b'$ and $(b', c') = 1$, then $c' \mid a'$.
  %\end{center}
  %Here we substitute $c' = p$, $a' = (a+b)$, $b' = (a - b)$.
  %So if $\text{gcd}(p, a+b) = 1$, then $p \mid (a-b)$.
\vfill
\newpage



% PROBLEM 3
\phantom{\quad} \vfill
\noindent
\textbf{Problem} (2.1.12) \\[4ex]
\emph{Solution.} \\[2ex]
  We want to show that $19 \not \, \mid 4n^2 + 4$ for any 
  $n \in \mathbb{Z}$.

  Theorem 2.12 tells us that $x^2 \equiv -1 \mod p$ for
  a given prime $p$ if and only if 
  $p = 2$ or $p \equiv 1 \mod 4$.

  Since $19 \equiv 3 \not \equiv 1 \mod 4$, we see that
  for any $n \in \mathbb{Z}$, 
  \begin{align*}
    n^2 &\not \equiv -1 \mod 19  \\
    n^2 + 1 &\not \equiv 0 \mod 19  \\
    4(n^2 + 1) &\not \equiv 0 \mod 19  \\
    4n^2 + 4 &\not \equiv 0 \mod 19  \\
    19 &\not \, \mid (4n^2 + 4).
  \end{align*}
  

  %Suppose we have a prime $p$ that satisfies that condition.
  %\begin{align*}
  %  n^2 &\equiv -1 \mod p \\
  %  n^2 + 1 &\equiv 0 \mod p \\
  %  4(n^2 + 1) &\equiv 0 \mod p & (*) \\
  %  4n^2 + 4 &\equiv 0 \mod p \\
  %  p &\mid (4n^2 + 4).
  %\end{align*}
  %Since each of the above are iff statements 
  %(except for (*), where $\text{gcd}(p, 4) = 1$ must hold
  %to go in the reverse direction),
  %we see that for $p \neq 2$, $p \mid (4n^2 + 4)$ for some
  %$n \in \mathbb{Z}_p$ if $p \equiv 1 \mod 4$.
  %But $19 \equiv 3 \not \equiv 1 \mod 4$, so there is no 
  %solution to the congruence $19 \mid (4n^2 + 4)$ for
  %any $n \in \mathbb{Z}_{19}$, and by extension in $\mathbb{Z}$. 
\vfill
\newpage



% PROBLEM 4
\phantom{\quad} \vfill
\noindent
\textbf{Problem} (2.1.20) \\[4ex]
\emph{Solution.} \\[2ex]
  We use Theorem 2.3(3).
  Let $m_1 = 2$, $m_2 = 3$, $m_3 = 7$, $x = n^7 - n$, 
  $n \in \mathbb{Z}$, and 
  $y = 0$.
  Then $n^7 - n \equiv 0 \mod m_i$ if and only if 
  $n^7 - n \equiv 0 \mod 42$.

  Using Fermat's Theorem (Theorem 2.7), we see that 
  \begin{align*}
    n^7 &\equiv n \mod 7 \\
    n^7 - n &\equiv 0 \mod 7.
  \end{align*}

  Likewise, we see that 
  \begin{align*}
    n^7     &= \left( n^3 \right)^2 n \\
    n^7     &\equiv n^2 \cdot n \mod 3 & \text{(Fermat)}\\
    n^7     &= n^3 \\
    n^7     & \equiv n \mod 3 & \text{(Fermat)}  \\
    n^7-n   & \equiv 0 \mod 3.
  \end{align*}

  And finally, we see that 
  \begin{align*}
    n^7 &= \left( n^2 \right)^3 n \\
    n^7    &\equiv n^3 \cdot n  \mod 2 & \text{(Fermat)}\\
    n^7    &= n^4 \\
    n^7    &= \left( n^2 \right)^2 \\
    n^7    &\equiv n^2  \mod 2 & \text{(Fermat)}\\
    n^7    &\equiv n \mod 2 & \text{(Fermat)} \\
    n^7-n  &\equiv 0 \mod 2.
  \end{align*}

  Since we have shown 
  for any $n \in \mathbb{Z}$,
  $n^7 - n \equiv 0 \mod m_i$ for $i = 1,2,3$, 
  Theorem 2.3(3) implies
  \[
  n^7 - n \equiv 0 \mod \text{lcm}(m_1, m_2, m_3) = 42, 
  \]
  which is equivalent to the statement 
  \[
    42 \mid n^7 - n
  \]
  for all $n \in \mathbb{Z}$.
\vfill
\newpage



% PROBLEM 5
\phantom{\quad} \vfill
\noindent
\textbf{Problem} (2.1.32) \\[4ex]
\emph{Solution.} \\[2ex]
  Suppose $m$ is odd.
  Clearly, the set 
  $\{ 1, 2, 3, \ldots, m\}$ is a complete
  residue system modulo $m$.
  Since $m$ is odd, $2 \not \, \mid m$, so
  $\text{gcd}(2, m) = 1$. 
  By Theorem 2.6, 
  $\{2 \cdot 1, 2 \cdot 2, 2 \cdot 3, \ldots 2 \cdot m \}
  = \{ 2, 4, 6, \ldots 2m\}$ is also a complete residue
  system modulo $m$.
\vfill
\newpage



% PROBLEM 6
\phantom{\quad} \vfill
\noindent
\textbf{Problem} (2.1.40) \\[4ex]
\emph{Solution.} \\[2ex]
  Let $m$ be odd.
  Clearly, $1, 2, 3, \ldots, m$ is a complete residue system.
  By definition, for any other complete residue system
  $r_1, r_2, r_3, \ldots, r_m$, each $r_i \equiv j_i$ for
  $1 \leq i \leq m$ and a unique $1 \leq j_i \leq m$.
  Thus, we may write the sum of elements as
  \begin{align*}
    \sum_{i = 1}^m r_i 
    &=
    \sum_{i = 1}^m (j_i + k_i m) & \text{where $j_i + k_i m = r_i, k_i \in \mathbb{Z}$} \\
    &=
      \sum_{i = 1}^m j_i +  \sum_{i = 1}^m k_i m  \\
    &=
      \sum_{j = 1}^m j +  \sum_{i = 1}^m k_i m  & \text{since each $j_i$ is unique}\\
    &=
      \frac{m(m+1)}{2} +  m \sum_{i = 1}^m k_i \\
    &=
      m \left( \frac{(m+1)}{2} +   \sum_{i = 1}^m k_i \right).
  \end{align*}
  Clearly, $\sum_{i = 1}^m k_i \in \mathbb{Z}$.
  Since $m$ is odd, $\frac{m+1}{2} \in \mathbb{Z}$, so 
  $\left( \frac{(m+1)}{2} +   \sum_{i = 1}^m k_i \right) \in \mathbb{Z}$ and 
  \[
    m \mathbin{\Big|} \sum_{i = 1}^m r_i
  \]
  for any complete residue system $r_1, r_2, \ldots, r_m$
  modulo $m$.

  Now we aim to prove the analogous result for a reduce residue system
  given any $m > 2$. 
  Let $U_m$ represent the group of units modulo $m$ whose elements are
  less than $m$. 
  A unit is an element of $\mathbb{Z}_m$ which has an inverse
  under multiplication. 
  By Theorem 2.9, an element $a \in Z_m$ will have an inverse 
  if $\gcd(a, m) = 1$.
  Clearly, $U_m$ a reduced residue system.

  First we aim to show that if $a \in U_m$, then $m-a \in U_m$.
  Assume $\gcd(a, m) = 1$. 
  By Theorem 1.9, $\gcd(a, m) = \gcd(-a, m) = \gcd(-a + mx, m )$ for any 
  $x \in \mathbb{Z}$. 
  Choose $x = 1$. 
  Thus, 
  \[
    1 = \gcd(a, m) = \gcd(m-a, m), 
  \]
  so $a \in U_m$ if and only if $m-a \in U_m$.
  (Note that the sum of these two numbers, $a$ and 
  $m -a$, is equal to $m$.)

  Next we aim to show that there is no such element
  $a' \in \mathbb{Z}$ where $a' = m - a'$ and 
  $\gcd(m, a') = 1$.
  If $a' = m - a'$, then $a' = \frac{m}{2}$.
  If $m$ is odd, no such number exists, so we 
  assume that $m$ is even. 
  Thus, $\frac{m}{2} \in \mathbb{N}$.
  Clearly,
  \[
    \gcd(a', m) 
    = \gcd \left( \frac{m}{2}, m \right) 
    = \frac{m}{2}.
  \]
  We see that $\frac{m}{2} = 1$ only when $m = 2$.
  For all other positive numbers, no such $a'$ exists.
  Thus, for every $a \in U_m$, where $m > 2$, there will be a 
  $b\in U_m$ such that $a \neq b$ and $a + b = m$.
  This also implies that $|U_m|$ is even for $m > 2$.

  Going on we will show that the reduced residue system
  $r_1, r_2, \ldots, r_s$, for $s \in \mathbb{N}$ where
  $r_i < m$ for $1 \leq i \leq s$ the elements sum 
  to a number divisible by $m$ if $m > 2$.
  Suppose $m > 2$.
  For a given $r_i \in U_m$, we have shown that there will
  be another $r_j \in U_m$, where $i \neq j$ and 
  $r_i + r_j = m$.
  Noting that $s = |U_m|$, we see that
  \begin{align*}
    \sum_{i=1}^s r_i &= \frac{s}{2} m,
  \end{align*} 
  where $\frac{s}{2} \in \mathbb{N}$ because the cardinality
  of $U_m$ is even.
  Therefore, 
  \[
    m \mid \sum_{i=1}^s r_i.
  \]

  Finally, we show that the above result
  holds for any reduced residue system.
  Consider the reduced residue system
  $q_1, q_2, \ldots, q_{s'}$.
  Since each reduced residue system has 
  the same number of elements ($\phi(m)$), 
  $s' = s$. 
  Each element $q_i$ is congruent to a 
  unique
  $r_j$ modulo $m$ for $1 \leq i,j \leq m$.
  Put another way, $q_i = r_j + km$ for some
  $k \in \mathbb{Z}$.
  Since the ordering of our set does not matter,
  we will say $q_i \equiv r_i \mod m$ for each
  $1 \leq i \leq m$.
  Now we have
  \begin{align*}
    \sum_{i=1}^s q_i &= \sum_{i=1}^s (r_i + k_i m) & (k_i \in \mathbb{Z}) \\
                     &= \sum_{i=1}^s r_i + \sum_{i=1}^s k_i m  \\
                     &= \frac{s}{2} m + m \sum_{i=1}^s k_i  \\
                     &= m \left( \frac{s}{2}  +  \sum_{i=1}^s k_i \right).
  \end{align*}
  Since $\frac{s}{2} \in \mathbb{Z}$ and $\sum_{i=1}^s k_i \in \mathbb{Z}$, 
  $\left( \frac{s}{2}  +  \sum_{i=1}^s k_i \right)  \in \mathbb{Z}$.
  By definition, $m \mid \sum_{i=1}^s q_i$.
  Since the choice of a reduced residue system 
  $q_1, q_2, \ldots, q_s$ 
  was arbitrary, 
  the condition that $m$ divides the sum of the elements of a 
  reduced residue system modulo $m$ holds if $m > 2$. 

  

\vfill
\newpage



% PROBLEM 7
\phantom{\quad} \vfill
\noindent
\textbf{Problem} (2.1.59) \\[4ex]
\emph{Solution.} \\[2ex]
  Let $p$ be prime.
  Suppose $p \mid a^2 + 2b^2$ for $a,b \in \mathbb{Z}$. 
  Since $p \not \, \mid a, b$; 
  $\text{gcd}(p, a) = \text{gcd}(p, b) = 1$.
  From Theorem 2.9, there exists $\bar{b} \in \mathbb{Z}_p$
  such that $\bar{b} b = b \bar{b} \equiv 1 \mod p$.
  We have
  \begin{align*}
    p &\mid a^2 + 2b^2 \\
    a^2 + 2b^2 &\equiv 0 \mod p \\
    a^2 &\equiv -2 b^2 \mod p \\
    a^2 \bar{b}^2 &\equiv -2 b^2 \bar{b}^2 \mod p \\
    \left( a \bar{b} \right)^2 &= -2 \left( b \bar{b} \right)^2 \\
    \left( a \bar{b} \right)^2 &\equiv -2 \mod p.
  \end{align*}
  Thus, for $x = a \bar{b}$, 
  \begin{align*}
    x^2 &= \left( a \bar{b} \right)^2 \\
        &\equiv -2 \mod p,
  \end{align*}
  so the congruence $x^2 \equiv -2 \mod p$ has a solution if
  $p \mid a^2 + 2b^2$ and $p \not \, \mid a,b$.
\vfill
\newpage



% PROBLEM 8
\phantom{\quad} \vfill
\noindent
\textbf{Problem} (2.2.6) \\[4ex]
\emph{Solution.} \\[2ex]
  %For these solutions we use Theorem 2.3(1), which states
  %that $ax \equiv ay \mod m$ if and only if 
  %$x \equiv y \mod \frac{m}{\text{gcd}(a,m)}$.
  We use Theorem 2.17 to determine the number of solutions.
	\begin{enumerate}
		\item[a.]
      A solution exists to the congruence 
      \[
        15 x \equiv 25 \mod 35
      \]
      because $g = \text{gcd}(15, 35) = 5, and 5 \mid 25$.
      Furthermore, there are $g = 5$ solutions modulus 35.
		\item[b.]
      No solution exists to the congruence
      \[
        15 x \equiv 24 \mod 35
      \]
      because $g = \text{gcd}(15, 35) = 5, and 5 \not \, \mid 24$.
		\item[c.]
      A solution exists to the congruence 
      \[
        15 x \equiv 0 \mod 35
      \]
      because $g = \text{gcd}(15, 35) = 5, and 5 \mid 0$.
      Furthermore, there are $g = 5$ solutions modulus 35.
	\end{enumerate}
\vfill
\newpage



% PROBLEM 9
\phantom{\quad} \vfill
\noindent
\textbf{Problem} (2.2.8) \\[4ex]
\emph{Solution.} \\[2ex]
   Let $p$ be an odd prime.
   We have
   \[
     x^2 \equiv 1 \mod p^{\alpha}.
   \]
   From this we see that 
   \begin{align}
     x^2 - 1 &\equiv 0 \mod p^{\alpha} \notag \\
     p^{\alpha} &\mid x^2 - 1 \notag \\
     p^{\alpha} &\mid (x+1)(x-1).\label{eq:2.2.8}
   \end{align}
   This also implies
   \[
     p \mid (x+1)(x-1).
   \]
   By Theorem 1.15, $p$ divides at least one of the
   factors on the right.
   %By Theorem 1.10, if $p \not \, \mid (x+1)$, then
   %$\text{gcd}(p, x+1) = 1$ so $p \mid (x-1)$.
   %Put another way, $p$ divides at least one of the two 
   %terms on the right. 

   Next we aim to show that $p$ cannot divide both 
   $x+1$ and $x-1$. 
   If it did, then by Theorem 1.1(3), 
   \begin{align*}
     p &\mid 1 \cdot (x + 1) + (-1)\cdot (x - 1) \\
     p &\mid 2.
   \end{align*}
   By Theorem 1.1(5), $p \mid 2$ implies $p \leq 2$. 
   Because 2 is the smallest prime number, $p \leq 2$ implies
   $p = 2$ since $p$ is prime. 
   But we assumed $p$ is odd, which is a contradiction
   since 2 is even.
   Thus, $p$ cannot divide one of these terms.
   Since the term that $p$ divides doesn't matter, we'll
   say that $p$ divides $x + 1$ but not $x - 1$.

   We know that $\text{gcd}(p, x-1) = 1$.
   By repeated application of Theorem 1.8, we see that
   $\gcd(p^{\alpha},x-1) = 1$.
   By Theorem 1.10 and equation \ref{eq:2.2.8}, 
   we see that $p^{\alpha} \mid (x+1)$.
   Since the term we assumed that $p$ could divide was arbitrary,
   this analysis also works if $p$ were to divide 
   $x -1$ but not $x + 1$.
\vfill
\newpage



% PROBLEM 10
\phantom{\quad} \vfill
\noindent
\textbf{Problem} (2.3.4) \\[4ex]
\emph{Solution.} \\[2ex]
  We are being asked to solve the system of linear congruences
  \begin{align*}
    x &\equiv 1 \mod 3 \\
    x &\equiv 2 \mod 4 \\
    x &\equiv 3 \mod 5.
  \end{align*}
  We start with the third congruence.
  The solution is $x = 3 + 5b$, for $b \in \mathbb{Z}$.
  Now we use this value in the second congruence to get
  \begin{align*}
    3 + 5b &\equiv 2 \mod 4 \\
    5 b &\equiv -1 \mod 4 \\
    b &\equiv -1 \mod 4 \\
    b &\equiv 3 \mod 4,
  \end{align*}
  so $b = 3 + 4c$, for $c \in \mathbb{Z}$. 
  We plug this in to get
  \begin{align*}
    x &= 3 + 5b  \\
      &= 3 + 5(3 + 4c) \\
      &= 3 + 15 + 20c \\
      &= 18 + 20c
  \end{align*}
   Next, we use this for the first congruence
  \begin{align*}
    18 + 20c &\equiv 1 \mod 3 \\
    2c &\equiv 1 \mod 3 \\
    2c &\equiv 4 \mod 3 \\
    c &\equiv 2 \mod 3 & \text{(because $\gcd(2, 3) = 1$, using Theorem 2.3(1))}
  \end{align*}
  so $c = 2 + 3d$, for $d \in \mathbb{Z}$.
 
  Finally, we have
  \begin{align*}
    x &= 18 + 20c \\
      &= 18 + 20(2 + 3d) \\
      &= 18 + 40 + 60d \\
      &= 58 + 60d,
  \end{align*}
  which is the solution to all three congruences.
\vfill
\newpage



% PROBLEM 11
\phantom{\quad} \vfill
\noindent
\textbf{Problem} (2.3.7) \\[4ex]
\emph{Solution.} \\[2ex]
  We aim to find whether or not the system of linear
  congruences
  \begin{align*}
    5x &\equiv 1 \mod 6 \\
    4x &\equiv 13 \mod 15
  \end{align*}
  has a solution, and if so what it is. 

  Using Theorem 2.1(5), we see that
  \[
    5x \equiv 1 \mod 6
  \]
  implies
  \begin{align}
    5x &\equiv 1 \mod 3  \notag \\
    2x &\equiv 4 \mod 3 \notag \\ 
    x  &\equiv 2 \mod 3, 
      & \text{(because $\gcd(2, 3) = 1$, using Theorem 2.3(1))} 
        \label{eq:2.3.7.1}
  \end{align}
  and 
  \[
    4x \equiv 13 \mod 15
  \]
  implies 
  \begin{align}
    4x &\equiv 13 \mod 3 \notag \\
    x &\equiv 1 \mod 3. \label{eq:2.3.7.2}
  \end{align}
  Clearly, equations \ref{eq:2.3.7.1} and \ref{eq:2.3.7.2} 
  are incompatible because $1 \not \equiv 2 \mod 3$. 
  Thus, the system of linear congruences is inconsistent
  and no solutions exist. 
\vfill
\newpage



% PROBLEM 12
\phantom{\quad} \vfill
\noindent
\textbf{Problem} (2.3.14) \\[4ex]
\emph{Solution.} \\[2ex]
  We aim to solve the system of congruences
  \begin{align*}
    x^3 + 2x - 3 &= 0 \mod 9 \\
    x^3 + 2x - 3 &= 0 \mod 5 \\
    x^3 + 2x - 3 &= 0 \mod 45.
  \end{align*}
  It is worth noting that by Theorem 2.16, 
  the first two congruences are implied by the third.

  We solve for the first two congruences and 
  use the Chinese remainder theorem to find solutions
  in $\mathbb{Z}_{45}$.

  Plugging in $x = 0, 1, 2, \ldots, 8$, we find that
  the soltions to the first congruence are
  $x = 1,2,6 \mod 9$.
  Likewise, by plugging in $x = 0, 1, 2, 3, 4, 5$, 
  we find that the solutions to the second congruence
  are $x = 1, 3 \mod 5$.

  Using Theorem 2.20, we see that $\gcd(5,9) = 1$ and
  $45 = 5 \cdot 9$, so the number of solutions 
  $N(45) = N(5)N(9) = 2 \cdot 3 = 6$ 
  for $f(x) = x^3 + 2x - 3$.
  All that needs to be done at this point is to use
  the Chinese remainder theorem to show six solutions
  in $\mathbb{Z}_{45}$.

  I will calculate the first solution for brevity.
  The algorithm for finding solutions is the same
  for the remaining five. 

  Let's find solutions to 
  \begin{align*}
    x &\equiv 1 \mod 9 \\
    x &\equiv 1 \mod 5
  \end{align*}
  We start with $x = 1 + 9a$ for $a \in \mathbb{Z}$
  and plug it into the second congruence
  \begin{align*}
    1 + 9a &\equiv 1 \mod 5 \\
    9a &\equiv 0 \mod 5 \\
    9a &\equiv 90 \mod 5 \\
    a &\equiv 10 \mod 5  
      & \text{(because $\gcd(9, 5) = 0$ using Theorem 2.3(1))} \\ 
    a &\equiv 0 \mod 5, 
  \end{align*}
  so $a = 0 + 5b$. 
  Plugging this into our first solution, we get
  \begin{align*}
    x &= 1 + 9a \\
      &= 1 + 9(0 + 5b) \\
      &= 1 + 45b,
  \end{align*}
  so $x \equiv 1 \mod 45$. 

  Finding the remaining five solutions is similarly tedious.
  The six solutions are 
  \begin{align*}
    x &\equiv 1 \mod 45 \\
    x &\equiv 6 \mod 45 \\
    x &\equiv 11 \mod 45 \\
    x &\equiv 28 \mod 45 \\
    x &\equiv 33 \mod 45 \\
    x &\equiv 38 \mod 45
  \end{align*}
\vfill
\newpage



% PROBLEM 13
\phantom{\quad} \vfill
\noindent
\textbf{Problem} (2.3.16) \\[4ex]
\emph{Solution.} \\[2ex]
  We aim to solve the congruence 
  \[
    x^3 - 9x^2 + 23x - 15 \equiv 0 \mod 503.
  \]
  We are given $503$ is prime and that
  \[
    x^3 - 9x^2 + 23x - 15 = (x-1)(x-3)(x-5).
  \]

  We see that 
  \begin{align*}
    x^3 - 9x^2 + 23x - 15 &\equiv 0 \mod 503 \\
    503 &\mid x^3 - 9x^2 + 23x - 15 \\
    503 &\mid (x-1)(x-3)(x-5).
  \end{align*}
  By Theorem 1.15, at least one of these terms is 
  divisible by 503. 

  From this we can find the following solutions
  \begin{align*}
    503 &\mid (x-1) 
      & 503 &\mid (x-3) 
      & 503 &\mid (x-5)\\ 
    x -1 &\equiv 0 \mod 503 
      & x -3 &\equiv 0 \mod 503 
      & x -5 &\equiv 0 \mod 503 \\
    x &\equiv 1 \mod 503
      & x &\equiv 3 \mod 503 
      & x &\equiv 5 \mod 503 
  \end{align*}
\vfill
\newpage



% PROBLEM 14
\phantom{\quad} \vfill
\noindent
\textbf{Problem} (2.3.17) \\[4ex]
\emph{Solution.} \\[2ex]
  We aim to solve the congruence 
  \[
    x^3 - 9x^2 + 23x - 15 \equiv 0 \mod 143.
  \]
  We see that $143 = 11 \cdot 13$ and 
  \begin{equation}
    \label{eq:2.3.17.1}
    x^3 - 9x^2 + 23x - 15 = (x-1)(x-3)(x-5).
  \end{equation}

  By Theorem 2.3(3), the first congruence
  is equivalent to the system of congruences
  \begin{align*}
    x^3 - 9x^2 + 23x - 15 &\equiv 0 \mod 11 \\
    x^3 - 9x^2 + 23x - 15 &\equiv 0 \mod 13.
  \end{align*}

  By using equation \ref{eq:2.3.17.1}, we see
  that the solutions to the above congruences are
  \begin{align*}
    x &\equiv 1 \mod 11 & x &\equiv 1 \mod 13 \\
    x &\equiv 3 \mod 11 & x &\equiv 3 \mod 13 \\
    x &\equiv 5 \mod 11 & x &\equiv 5 \mod 13.
  \end{align*}
  
  Now from Theorem 2.20, since $\gcd(11,13) = 1$
  and $11 \cdot 13 = 143$, 
  $N(143) = N(11)N(13) = 3 \cdot 3 = 9$ for 
  $f(x) = x^3 - 9x^2 + 23x - 15$, so we expect
  there to be 9 solutions to the original congruence.

  We can find each solution by constructing 
  9 pairs of linear congruences and finding the
  solution with the Chinese Remainder Theorem.
  
  I will solve an arbitrary pair of linear congruences.
  The remaining 8 are equally tedious.

  Consider the following congruences
  \begin{align*}
    x &\equiv 3 \mod 11 \\
    x &\equiv 5 \mod 13.
  \end{align*}
  The solution to the first congruence is 
  $x = 3 + 11a$ for some $a \in \mathbb{Z}$.
  Now we plug this into the second congruence
  to get
  \begin{align*}
    3 + 11a &\equiv 5 \mod 13 \\
    11a &\equiv 2 \mod 13 \\
    -2a &\equiv 2 \mod 13 \\
    2a &\equiv -2 \mod 13 \\
    a &\equiv -1 \mod 13 & \text{(because $\gcd(2,13) = 1$ using Theorem 2.3(1))} \\ 
    a &\equiv 12 \mod 13.
  \end{align*}
  so $a = 12 + 13b$, for $b \in \mathbb{Z}$.
  Plugging this into $x$ we get
  \begin{align*}
    x &= 3 + 11a \\
      &= 3 + 11(12 + 13b) \\
      &= 3 + 132 + 143b \\
      &= 135 + 143b,
  \end{align*}
  so $x \equiv 135 \mod 143$ is one solution.

  The full set of solutions is
  \begin{align*}
    x &\equiv 1 \mod 143 \\
    x &\equiv 3 \mod 143 \\
    x &\equiv 5 \mod 143 \\
    x &\equiv 14 \mod 143 \\
    x &\equiv 16 \mod 143 \\
    x &\equiv 27 \mod 143 \\
    x &\equiv 122 \mod 143 \\
    x &\equiv 133 \mod 143 \\
    x &\equiv 135 \mod 143
  \end{align*}

\vfill
\newpage



% PROBLEM 15
\phantom{\quad} \vfill
\noindent
\textbf{Problem} (2.3.20) \\[4ex]
\emph{Solution.} \\[2ex]
  For $m_1, m_2 \in \mathbb{Z}^+$ fixed, and 
  $a_1, a_2 \in \mathbb{Z}$ arbitrary, we have
  \begin{align*}
    x &\equiv a_1 \mod m_1 \\
    x &\equiv a_2 \mod m_2.
  \end{align*}

  Let us try to find a solution to the above
  system of linear congruences.
  We start with the general solution to the first
  congruence:
  $x = a_1 + m_1 \cdot k$, 
  for $k \in \mathbb{Z}$.
  Let $g = \gcd(m_1, m_2)$.
  Next we plug that into the second congruence
  \begin{align*}
    a_1 + m_1 \cdot k &\equiv a_2 \mod m_2 \\
    k \cdot m_1 &\equiv a_2 - a_1 \mod m_2 \\
    k \frac{m_1}{g} &\equiv \frac{a_2 - a_1}{g} \mod \frac{m_2}{g} 
      & \text{by Theorem 2.3(1)} \quad (*)\\
    k \frac{m_1}{g} 
      \overline{\left( \frac{m_1}{g} \right)} 
      &\equiv 
      \frac{a_2 - a_1}{g} 
      \overline{\left( \frac{m_1}{g} \right)} 
      \mod \frac{m_2}{g} & (**) \\
    k &\equiv 
      \frac{a_2 - a_1}{g} 
      \overline{\left( \frac{m_1}{g} \right)} 
      \mod \frac{m_2}{g},
  \end{align*}
  so $k = \frac{a_2 - a_1}{g}  
          \overline{\left( \frac{m_1}{g} \right)}
          + \frac{m_2}{g} \cdot o$, 
  for $o \in \mathbb{Z}$.

  We use (*) to note $a_2 - a_1$ must be divisible by
  $g$, or $a_1 \equiv a_2 \mod g$. 
  We use (**) to note that 
  $\gcd \left( \frac{m_1}{g}, \frac{m_2}{g} \right) = 1$ by 
  Theorem 1.7, so $\frac{m_1}{g}$ has a unique inverse in 
  $\mathbb{Z}_{\frac{m_2}{g}}$ by Theorem 2.9, which we
  denote as $\overline{\left( \frac{m_1}{g} \right)}$.

  Each of the steps up to this point has been if and only
  statements, so a solution to the system of linear
  congruences 
  \begin{align*}
    x &\equiv a_1 \mod m_1 \\
    x &\equiv a_2 \mod m_2 
  \end{align*}
  exists if and only if $a_1 \equiv a_2 \mod g$, for 
  $g = \gcd(m_1, m_2)$. 

  Now we can plug $k$ back into $x$ to get
  \begin{align*}
    x &= a_1 + m_1 \cdot k \\
    x &= a_1 + m_1
        \left(
          \frac{a_2 - a_1}{g}  
          \overline{\left( \frac{m_1}{g} \right)}
          + \frac{m_2}{g} \cdot o
        \right) \\
    x &= a_1 
        + m_1 
          \frac{a_2 - a_1}{g}  
          \overline{\left( \frac{m_1}{g} \right)}
        + \frac{m_1 m_2}{g} \cdot o \\
    x &= a_1 
        + m_1 
          \frac{a_2 - a_1}{g}  
          \overline{\left( \frac{m_1}{g} \right)}
          + \text{lcm}(m_1, m_2) \cdot o  & (***) \\
    x &\equiv 
        a_1 
        + m_1 
          \frac{a_2 - a_1}{g}  
          \overline{\left( \frac{m_1}{g} \right)}
        \mod \text{lcm}(m_1, m_2),
  \end{align*}
  where (***) is using Theorem 1.13, which states that 
  \begin{align*}
    \text{lcm}(m_1, m_2) \gcd(m_2, m_2) &= |m_1 m_2|  \\
    \text{lcm}(m_1, m_2)  &= \frac{|m_1 m_2|}{\gcd(m_1, m_2)}  \\
    \text{lcm}(m_1, m_2)  &= \frac{m_1 m_2}{g} & \text{(since $m_1, m_2 \in \mathbb{Z}^+$)}
  \end{align*}
  Clearly, if a solution to our original set of linear congruences
  exists, the solution will be a single congruence class
  (recall that inverses are unique in $\mathbb{Z}_n$ if they exist) 
  modulo $\text{lcm}(m_1, m_2)$. 
  Thus, the solution is unique 
  modulo $\text{lcm}(m_1, m_2)$. 
\vfill
\end{document}
