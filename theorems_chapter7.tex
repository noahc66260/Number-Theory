\documentclass{amsart}

\usepackage{amsthm}
\usepackage{amsmath}
\usepackage{fullpage}

\newtheorem{theorem}{Theorem}
\newtheorem{lemma}{Lemma}
\theoremstyle{definition}
\newtheorem*{definition*}{Definition}
\newtheorem*{note*}{Note}
\begin{document}

\section*{Chapter 7}

\begin{theorem}[7.1]
  If
  $\langle a_0, a_1, \cdots, a_j \rangle 
  = \langle b_0, b_1, \cdots, b_j \rangle$
  where these finite continued fractions are simple,
  and if $a_j > 1$ and $b_n > 1$, then $j = n$ and
  $a_i = b_i$ for $i = 0, 1, \cdots, n$.
\end{theorem}

\begin{theorem}[7.2]
  Any finite simple continued fraction represents a rational 
  number. 
  Conversely, any rational number can be expressed as a finite
  simple continued fraction, and in exactly two ways. 
\end{theorem}

\begin{note*}
  Suppose $u_0 / u_1 \in \mathbb{Q}$.
  Then
  $\frac{u_0}{u_1} 
  = \langle a_0, a_1, \cdots a_j \rangle
  = \langle a_0, a_1, \cdots a_j - 1, 1 \rangle$.
  Those are its two sole representations.
\end{note*}

\begin{definition*}
  Let $a_0, a_1, a_2, \ldots$ be an infinite sequence
  of integers, all positive except perhaps $a_0$.
  We define two sequences of integers
  $\{ h_n \}$ and $\{ k_n \}$ inductively as follows:
  \begin{align*}
    h_{-2} = 0, h_{-1} = 1, h_i &= a_i h_{i-1} + h_{i-2}, 
      \quad \text{for $i \geq 0$} \\
      k_{-2} = 1, k_{-1} = 0, k_i &= a_i k_{i-1} + k_{i-2}, 
      \quad \text{for $i \geq 0$}
  \end{align*}
  We note that $1 = k_0 < k_1 < k_2 < k_3 < \cdots < k_n < \cdots$.
\end{definition*}

\begin{theorem}
  For any positive real number $x$,
  \[
    \langle a_0, a_1, \cdots, a_{n-1}, x \rangle
    = \frac{x h_{n-1} + h_{n-2}}{x k_{n-1} + k_{n-2}}.
 \]
\end{theorem}

\begin{theorem}
  If we define $r_n = \langle a_0, a_1, \cdots, a_n \rangle$
  for all integers $n \geq 0$, then 
  $r_n = \frac{h_n}{k_n}$.
\end{theorem}

\begin{theorem}
  The equations
  \[
    h_i k_{i-1} - h_{i-1} k_i = (-1)^{i-1}
    \quad
    \text{and}
    \quad
    r_i - r_{i-1} = \frac{(-1)^{i-1}}{k_i k_{i-1}}
  \]
  hold for $i \geq 1$.
  The identities 
  \[
    h_i k_{i-2} - h_{i-2} k_i = (-1)^{i} a_i
    \quad
    \text{and}
    \quad
    r_i - r_{i-2} = \frac{(-1)^{i-1}}{k_i k_{i-2}}
  \]
  hold for $i \geq 1$.
  The fraction $h_i / k_i$ is reduced, that is
  $(h_i, k_i) = 1$.
\end{theorem}

\begin{theorem}
  The values $r_n$ defined in Theorem 7.4 satisfy
  the infinite chan of inequalities
  $r_0 < r_2 < r_4 < r_6 < \cdots < r_7 < r_5 < r_3 < r_1$.
  Stated in words, the $r_n$ with even subscripts form an 
  increasing sequence, those with odd subscripts form a decreasing
  sequence, and every $r_{2n}$ is less than every
  $r_{2j-1}$.
  Furthermore, $\lim_{n \to \infty} r_n$ exists, and
  for every $j \geq 0$, $r_{2j} < \lim_{n \to \infty} r_n < r_{2j+1}$.
\end{theorem}

\begin{definition*}
  An infinite sequence 
  $a_0, a_1, a_2, \cdots$
  of integers, all positive except perhaps for $a_0$, determines
  an infinite simple continued fraction 
  $\langle a_0, a_1, a_2, \cdots \rangle$.
  The value of 
  $\langle a_0, a_1, a_2, \cdots \rangle$
  is defined to be 
  $\lim_{n \to \infty} \langle a_0, a_1, a_2, \cdots, a_n \rangle$.
\end{definition*}

\begin{theorem}
  The value of any infinite simple continued fraction
  $\langle a_0, a_1, a_2, \cdots \rangle$ is irrational.
\end{theorem}

\begin{theorem}
  Two distinct infinite simple continued fractions converge to different
  values.
\end{theorem}

\begin{lemma}
  Let $\theta = \langle a_0, a_1, a_2, \cdots \rangle$
  be a simple continued fraction.
  Then $a_0 = [\theta]$.
  Furthermore, if $\theta_1$ denotes
  $\langle a_1, a_2, a_3, \cdots \rangle$, then
  $\theta = a_0 + 1 / \theta_1$.
\end{lemma}

\begin{note*}
  To construct the continued fraction of an irrational number:
  \begin{align*}
    \xi_0 &= \xi \\
    a_i &= [\xi_i] \\
    \xi_{i+1} &= \frac{1}{\xi_i - a_i}.
  \end{align*}
  Note that this implies $\xi_i = a_i + \frac{1}{\xi_{i+1}}$.
  Now we get
  \begin{align*}
    \xi &= \xi_0 = a_0 + \frac{1}{\xi_1} = \langle a_0, \xi_1 \rangle \\
        &= \left \langle a_0, a_1 + \frac{1}{\xi_2} \right \rangle
           = \langle a_0, a_1, \xi_2 \rangle \\
        &= \qquad \vdots \\
        &= \left \langle 
    a_0, a_1, a_2, \cdots, a_{m-2}, a_{m-1} + \frac{1}{\xi_m} 
           \right \rangle 
           = \langle a_0, a_1, \cdots, a_{m-1}, \xi_m \rangle,
  \end{align*}
  so 
  \begin{align*}
    \xi = \langle a_0, a_1, \cdots, a_{n-1}, \xi_n \rangle
    &= \frac{\xi_n h_{n-1} + h_{n-2}}{\xi_n k_{n-1} + k_{n-2}}.
  \end{align*}
\end{note*}

\begin{theorem}
  Any irrational number $\xi$ is uniquely expressible, 
  by the procedure gave in the previous note, as an infinite
  simple continued fraction
  $\langle a_0, a_1, a_2, \cdots \rangle$.
  Conversely, any such continued fraction determined by integers
  $a_i$ that are positive for all $i > 0$ represents an 
  irrational number, $\xi$.
  The finite simple continued fraction 
  $\langle a_0, a_1, \cdots, a_n \rangle$ has the rational value
  $\frac{h_n}{k_n} = r_n$, and is called the nth convergent to 
  $\xi$.
  Even convergents form a monotonically increasing sequence with
  $\xi$ as the limit.
  Similarly, odd convergents form a monotonically decreasing sequence
  tending to $\xi$.
  The denominators $k_n$ of the convergents are an increasing
  sequence of positive integers for $n > 0$.
  Finally, with $\xi_i$ defined as in our previous note,
  we have 
  $\langle a_0, a_1, \cdots \rangle 
  = \langle a_0, a_1, \cdots, a_{n-1}, \xi_n \rangle$
  and 
  $\xi_n = \langle a_n, a_{n+1}, a_{n+2}, \cdots \rangle$.
\end{theorem}

\begin{theorem}
  We have for any $n \geq 0$,
  \begin{align*}
    \left| \xi - \frac{h_n}{k_n} \right|
    < 
    \frac{1}{k_n k_{n+1}}
    \quad
    \text{and}
    \quad
    | \xi k_n - h_n | < \frac{1}{k_{n+1}}
  \end{align*}
\end{theorem}

\begin{theorem}
  The convergents $h_n / k_n$ are successively closer to $\xi$, that is
  \begin{align*}
    \left| \xi - \frac{h_n}{k_n} \right|
    < 
    \left| \xi - \frac{h_{n-1}}{k_{n-1}} \right|,
  \end{align*}
  in fact the stronger inequality
  \begin{align*}
    \left| \xi k_{n} - h_{n} \right|
    < 
    \left| \xi k_{n-1} - h_{n-1} \right|
  \end{align*}
  holds.
\end{theorem}

\begin{theorem}
  If $a/b$ is a rational number with positive denominator such that 
  $|\xi - a/b | < |\xi - h_n/k_n|$ for some $n \geq 1$, then
  $b > k_n$.
  In fact if $|\xi b - a| < |\xi k_n - h_n|$ for some
  $n \geq 0$, then $b \geq k_{n+1}$.
\end{theorem}

\begin{theorem}
  Let $\xi$ denote any irrational number.
  If there is a rational number $a/b$ with $b \geq 1$ such that
  \[
    \left| \xi - \frac{a}{b} \right| < \frac{1}{2b^2},
  \]
  then $a/b$ equals one of the convergents of the simple continued fraction 
  expansion of $\xi$.
\end{theorem}

\begin{theorem}
  The nth convergent of $1/x$ is the reciprocal of the 
  $(n-1)$st convergetn of $x$ if $x$ is any real number $> 1$.
\end{theorem}

\begin{theorem}
  Hurwitz. Given any irrational number $\xi$, there exist
  infinitely many rational numbers $h/k$ such that
  \begin{align*}
    \left| \xi - \frac{h}{k} \right| < \frac{1}{\sqrt{5}k^2}.
  \end{align*}
\end{theorem}

\begin{theorem}
  The constant $\sqrt{5}$ in the preceding theorem is the best
  possible.
  In other words, the theorem does not hold if $\sqrt{5}$ is
  replaced by a larger number.
\end{theorem}

\end{document}
